\section{Multigrid Background\label{sec:background}}
\markboth{\textsc{MLD2P4 User's and Reference Guide}}
         {\textsc{\ref{sec:background} Multigrid Background}}

Multigrid preconditioners, coupled with Krylov iterative
solvers, are widely used in the parallel solution of large and sparse linear systems,
because of their optimality in the solution of linear systems arising from the
discretization of scalar elliptic Partial Differential Equations (PDEs) on regular grids.
Optimality, also known as algorithmic scalability, is the property 
of having a computational cost per iteration that depends linearly on
the problem size, and a convergence rate that is independent of the problem size.

Multigrid preconditioners are based on a recursive application of a two-grid process
consisting of smoother iterations and a coarse-space (or coarse-level) correction.
The smoothers may be either basic iterative methods, such as the Jacobi and Gauss-Seidel ones,
or more complex subspace-correction methods, such as the Schwarz ones.
The coarse-space correction consists of solving, in an appropriately chosen
coarse space, the residual equation associated with the approximate solution computed
by the smoother, and of using the solution of this equation to correct the
previous approximation. The transfer of information between the original
(fine) space and the coarse one is performed by using suitable restriction and
prolongation operators. The construction of the coarse space and the corresponding
transfer operators is carried out by applying a so-called coarsening algorithm to the system
matrix. Two main approaches can be used to perform coarsening: the geometric approach,
which exploits the knowledge of some physical grid associated with the matrix
and requires the user to define transfer operators from the fine
to the coarse level and vice versa, and the algebraic approach, which builds
the coarse-space correction and the associate transfer operators using only matrix
information. The first approach may be difficult when the system comes from
discretizations on complex geometries;
furthermore, ad hoc one-level smoothers may be required to get an efficient
interplay between fine and coarse levels, e.g., when matrices with highly varying coefficients
are considered. The second approach performs a fully automatic coarsening and enforces the
interplay between fine and coarse level by suitably choosing the coarse space and
the coarse-to-fine interpolation (see, e.g., \cite{Briggs2000,Stuben_01,dd2_96} for details.)

MLD2P4 uses a pure algebraic approach, based on the smoothed 
aggregation algorithm \cite{BREZINA_VANEK,VANEK_MANDEL_BREZINA},
for building the sequence of coarse matrices and transfer operators,
starting from the original one.
A decoupled version of this algorithm is implemented, where the smoothed
aggregation is applied locally to each submatrix \cite{TUMINARO_TONG}.
A brief description of the AMG preconditioners implemented in MLD2P4 is given in 
Sections~\ref{sec:multilevel}-\ref{sec:smoothers}. For further details the reader
is referred to \cite{para_04,aaecc_07,apnum_07,MLD2P4_TOMS}.

We note that optimal multigrid preconditioners do not necessarily correspond
to minimum execution times in a parallel setting. Indeed, to obtain effective parallel
multigrid preconditioners, a tradeoff between the optimality and the cost of building and
applying the smoothers and the coarse-space corrections must be achieved. Effective
parallel preconditioners require algorithmic scalability to be coupled with implementation
scalability, i.e., a computational cost per iteration which remains (almost) constant as
the number of parallel processors increases.


\subsection{AMG preconditioners\label{sec:multilevel}}

In order to describe the AMG preconditioners available in MLD2P4, we consider a
linear system
\begin{equation}
Ax=b, \label{eq:system}
\end{equation}
where $A=(a_{ij}) \in \mathbb{R}^{n \times n}$ is a nonsingular sparse matrix;
for ease of presentation we assume $A$ is real, but the
results are valid for the complex case as well. 

Let us assume as finest index space the set of row (column) indices of $A$, i.e.,
$\Omega = \{1, 2, \ldots, n\}$. 
Any algebraic multilevel preconditioners implemented in MLD2P4 generates
a hierarchy of index spaces and a corresponding hierarchy of matrices,
\[ \Omega^1 \equiv \Omega \supset \Omega^2 \supset \ldots \supset \Omega^{nlev},
\quad A^1 \equiv A, A^2, \ldots, A^{nlev}, \]
by using the information contained in $A$, without assuming any
knowledge of the geometry of the problem from which $A$ originates.
A vector space $\mathbb{R}^{n_{k}}$ is associated with $\Omega^k$,
where $n_k$ is the size of $\Omega^k$.
For all $k < nlev$, a restriction operator and a prolongation one are built,
which connect two levels $k$ and $k+1$:
$$
    P^k \in \mathbb{R}^{n_k \times n_{k+1}}, \quad 
    R^k \in \mathbb{R}^{n_{k+1}\times n_k};
$$
%\[ 
%   P^k:  \mathbb{R}^{n_{k+1}} \longrightarrow \mathbb{R}^{n_k}, \quad
%   R^k: \mathbb{R}^{n_k} \longrightarrow \mathbb{R}^{n_{k+1}};
%\]
the matrix $A^{k+1}$ is computed by using the previous operators according
to the Galerkin approach, i.e.,
$$
  A^{k+1}=R^kA^kP^k.
$$
$R^k=(P^k)^T$ in the current implementation of MLD2P4.
A smoother with iteration matrix $M^k$ is set up at each level $k < nlev$, and a solver
is set up at the coarsest level, so that they are ready for application 
(for example, setting up a solver based on the $LU$ factorization means computing
and storing the $L$ and $U$ factors). The construction of the hierachy of AMG components
described so far corresponds to the so-called build phase of the preconditioner.

\begin{figure}[t]
\begin{center} 
\framebox{
\begin{minipage}{.85\textwidth}
\begin{tabbing}
\quad \=\quad \=\quad \=\quad \\[-3mm]
procedure V-cycle$\left(k,A^k,b^k,u^k\right)$ \\[2mm]
\>if $\left(k \ne nlev \right)$ then \\[1mm]
\>\> $u^k = u^k + M^k \left(b^k - A^k u^k\right)$ \\[1mm]
\>\> $b^{k+1} = R^{k+1}\left(b^k - A^k u^k\right)$ \\[1mm]
\>\> $u^{k+1} =$ V-cycle$\left(k+1,A^{k+1},b^{k+1},0\right)$ \\[1mm]
\>\> $u^k = u^k + P^{k+1} u^{k+1}$ \\[1mm]
\>\> $u^k = u^k + M^k \left(b^k - A^k u^k\right)$ \\[1mm]
\>else \\[1mm]
\>\> $u^k = \left(A^k\right)^{-1} b^k$\\[1mm]
\>endif \\[1mm]
\>return $u^k$ \\[1mm]
end
\end{tabbing}
\end{minipage}
}
\caption{Application phase of a V-cycle preconditioner.\label{fig:application_alg}}
\end{center}
\end{figure}

The components produced in the build phase may be combined in several ways
to obtain different multilevel preconditioners;
this is  done in the application phase, i.e., in the computation of a vector
of type $w=B^{-1}v$, where $B$ denotes the preconditioner, usually within an iteration
of a Krylov solver \cite{Saad_book}. An example of such a combination, known as
V-cycle, is given in Figure~\ref{fig:application_alg}. In this case, a single iteration
of the same smoother is used before and after the the recursive call to the V-cycle (i.e.,
in the pre-smoothing and post-smoothing phases); however, different choices can be
performed. Other cycles can be defined; in MLD2P4, we implemented the standard V-cycle
and W-cycle~\cite{Briggs2000}, and a version of the K-cycle described in~\cite{Notay2008}. 


\subsection{Smoothed Aggregation\label{sec:aggregation}}

In order to define the prolongator $P^k$, used to compute
the coarse-level matrix $A^{k+1}$, MLD2P4 uses the smoothed aggregation
algorithm described in \cite{BREZINA_VANEK,VANEK_MANDEL_BREZINA}.
The basic idea of this algorithm is to build a coarse set of indices
$\Omega^{k+1}$ by suitably grouping the indices of $\Omega^k$ into disjoint
subsets (aggregates), and to define the coarse-to-fine space transfer operator
$P^k$ by applying a suitable smoother to a simple piecewise constant
prolongation operator, with the aim of improving the quality of the coarse-space correction.

Three main steps can be identified in the smoothed aggregation procedure:
\begin{enumerate}
        \item aggregation of the indices set $\Omega^k$, to obtain $\Omega^{k+1}$;
        \item construction of the prolongator $P^k$;
        \item application of $P^k$ and $R^k=(P^k)^T$ to build $A^{k+1}$.
\end{enumerate}
 
In order to perform the coarsening step, the smoothed aggregation algorithm
described in~\cite{VANEK_MANDEL_BREZINA} is used. In this algorithm,
each index in $\Omega^{k+1}$ corresponds to an aggregate of $\Omega^k$,
consisting of a suitably chosen index $j$ and of the indices $i$ that are strongly
coupled to $j$, i.e.,
 $$
    |a_{ij}^k| > \theta \sqrt{|a_{ii}^ka_{jj}^k|},
$$
for a given $\theta \in [0,1]$. Since this algorithm has a sequential nature, a decoupled
version of it is applied, where each processor $i$ independently executes
the algorithm on the set of indices assigned to it in the initial data
distribution. This version is embarrassingly parallel, since it does not require any data 
communication. On the other hand, it may produce some non-uniform aggregates
and is strongly dependent on the number of processors and on the initial partitioning
of the matrix $A$. Nevertheless, this parall algorithm has been chosen for
MLD2P4, since it has been shown to produce good results in practice
\cite{aaecc_07,apnum_07,TUMINARO_TONG}.

The prolongator $P^k$ is built starting from a tentative prolongator
$\bar{P}^k \in \mathbb{R}^{n_k \times n_{k+1}}$, defined as
$$
\bar{P}^k =(\bar{p}_{ij}^k), \quad  \bar{p}_{ij}^k = 
\left\{ \begin{array}{ll}
1 & \quad \mbox{if} \; i \in \Omega^k_j, \\
0 & \quad \mbox{otherwise},
\end{array} \right.
\label{eq:tent_prol}
$$
where $\Omega^k_j$ is the aggregate of $\Omega^k$
corresponding to the index $j \in \Omega^{k+1}$.
$P^k$ is obtained by applying to $\bar{P}^k$ a smoother
$S^k \in \mathbb{R}^{n_k \times n_k}$:
$$
P^k = S^k \bar{P}^k,
$$
in order to remove nonsmooth components from the range of the prolongator,
and hence to improve the convergence properties of the multi-level
method~\cite{BREZINA_VANEK,Stuben_01}.
A simple choice for $S^k$ is the damped Jacobi smoother:
$$
S^k = I - \omega^k (D^k)^{-1} A^k , 
$$
where $D^k$ is the diagonal matrix with the same diagonal entries as $A^k$,
and $\omega^k$ is an approximation of $4/(3\rho^k)$, where
$\rho^k$ is the spectral radius of $(D^k)^{-1}A^k$.
computed by using some estimate of the spectral radius of $(D^k)^{-1}A^k$
\cite{BREZINA_VANEK}.

\subsection{Smoothers and coarsest-level solvers\label{sec:smoothers}}

The smoothers implemented in MLD2P4 include the Jacobi and block-Jacobi methods,
a hybrid version of the forward and backward Gauss-Seidel methods, and the
additive Schwarz (AS) ones (see, e.g., \cite{Saad_book,dd2_96}). 

The hybrid Gauss-Seidel
version is considered because the original Gauss-Seidel method is inherently sequential.
At each iteration of the hybrid version, each parallel process uses the most recent values
of its own local variables and the values of the non-local variables computed at the
previous iteration, obtained by exchanging data with other processes before
the beginning of the current iteration.

In the AS methods, the index space $\Omega^k$ is divided into $m_k$
subsets $\Omega^k_i$ of size $n_{k,i}$,  possibly
overlapping. For each $i$ we consider the restriction
operator $R_i^k \in \mathbb{R}^{n_{k,i} \times n_k}$
% $R_i^k: \mathbb{R}^{n_k} \longrightarrow \mathbb{R}^{n_{k,i}}$
that maps a vector $x^k$ to the vector $x_i^k$ made of the components of $x^k$
with indices in $\Omega^k_i$, and the prolongation operator
$P^k_i = (R_i^k)^T$. These operators are then  used to build
$A_i^k=R_i^kA^kP_i^k$, which is the restriction of $A^k$ to the index
space $\Omega^k_i$.
The classical AS preconditioner $M^k_{AS}$ is defined as
\[
    ( M^k_{AS} )^{-1} = \sum_{i=1}^{m_k} P_i^k (A_i^k)^{-1} R_i^{k},
\]
where $A_i^k$ is supposed to be nonsingular. We observe that an approximate
inverse of $A_i^k$ is usually considered instead of $(A_i^k)^{-1}$.
The setup of $S^k_{AS}$ during the multilevel build phase
involves
\begin{itemize}
  \item the definition of the index subspaces $\Omega_i^k$ and of the corresponding 
  operators $R_i^k$ (and $P_i^k$);
  \item the computation of the submatrices $A_i^k$;
  \item the computation of their inverses (usually approximated
    through  some form  of incomplete factorization).
\end{itemize}
The computation of $z^k=M^k_{AS}w^k$, with $w^k \in \mathbb{R}^{n_k}$, during the
multilevel application phase, requires
\begin{itemize}
	\item the restriction of $w^k$ to the subspaces $\mathbb{R}^{n_{k,i}}$,
	  i.e.\ $w_i^k = R_i^{k} w^k$;
	\item the computation of the vectors $z_i^k=(A_i^k)^{-1} w_i^k$;
	\item the prolongation and the sum of the previous vectors,
    i.e.\ $z^k = \sum_{i=1}^{m_k} P_i^k z_i^k$.
\end{itemize}
Variants of the classical AS method, which use modifications of the
restriction and prolongation operators, are also implemented in MLD2P4.
Among them, the Restricted AS (RAS) preconditioner usually
outperforms the classical AS preconditioner in terms of convergence
rate and of computation and communication time on parallel distributed-memory
computers, and is therefore the most widely used among the AS
preconditioners~\cite{CAI_SARKIS}. 

Direct solvers based on sparse LU factorizations, implemented in the
third party libraries reported in Section~\ref{sec:third_party}, can be applied
as coarsest-level solvers by MLD2P4. Native inexact solvers based on
incomplete LU factorizations, as well as Jacobi, hybrid (forward) Gauss-Seidel,
and block Jacobi preconditioners are also available. Direct solvers usually
lead to more effective preconditioners in terms of algorithmic scalability;
however, this does not guarantee parallel efficiency.

%%% Local Variables: 
%%% mode: latex
%%% TeX-master: "userguide"
%%% End: 
